\section{Inledning}

Detta arbete jämför olika ramverk för uppskalning av agil programutveckling. Dessa ramverk används för att applicera traditionella agila metoder på större projekt och grupper.

\newpage

\section{Bakgund} 	

%TODO: Teoretisk bakgrund och genomgång av problemuppläggningen

I detta stycke behandlas den tekniska bakgrunden till arbetet. Eventuella behov av bakgrunds finns här.

\subsection{Agil utveckling}

Agil utveckling är en metod, eller en samling principer, för programutveckling. Principerna bygger på att bryta ner en stor helhet i små mindre självständiga delar, som man sedan utvecklar i skilda etapper, ofta kallade "Sprints".
Mellan varje sprint finns möjlighet för kunden och utvecklarna att komma med förändringsförslag och kommentera förra sprintens resultat. Varje sprint ska producera en fungerande helhet som läggs till huvudprodukten. Centrala begrepp och principer inom agil programutveckling är transparens, flexibilitet samt inkrementell och iterativ utveckling. Man värdesätter flexibilitet och kommunkation med kunden mer än att noggrannt definiera en process och sedan följa den. \cite{agile_manifesto}

\subsection{Skalning av agil utveckling}


\subsection{Large-Scale Scrum}


\subsection{Scaled Agile Framework}


\subsection{Disciplined Agile}


	
\newpage

\section{Syfte, avgränsningar, material och metoder}


\subsection{Syfte}
%TODO: Syftet med arbetet

Syftet med arbetet är att klargöra vilka skillnader det finns mellan olika ramverk för skalning av agil utveckling. Tekniska skillnader i användningen och definitionerna av ramverken pekas ut och analyseras.
Tyngdpunkten ligger på att redogöra för vilka situationer olika ramverk lämpar sig bättre än andra, och att ställa ramverkens styrkor och svagheter mot varandra.



\subsection{Avgränsning}
%TODO: Avgränsa arbetet

Ramverken som jämförs i detta arbete är begränsade till Large Scale Scrum (LeSS), Scaled Agile Framework (SAFe) samt Disciplined Agile Delivery (DAD).
Dess ramverk är allmänt använda i olika instanser.




\subsection{Material och metoder}
%TODO: Beskriv arbtessättet, speciellt för informationssökningens del
Arbetets två delar använder sig av två olika sorters material.

Jämförelsen av ramverkens tekniska speficikationer och principer sker på basis av tillgänglig dokumentation. Böcker och tekniska speficikationer används.

%TODO Sätt in matris med tillgänglig dokumentation för de olika ramverken. T.ex. Finns det en skriven bok, tekniska specifikationer etc.

Ramverkens styrkor, svagheter och användingsmöjligheter jämför primärt på basis av fallstudier gjorde av företag som använt sig av ramverken i praktiken.

%todo Fräsig matris/statistik på fallstudier per ramverk( och kvalitén på dessa?)


Arbetets analys och slutsatser är starkt bundna av tillgången till material och på kvalitén av det tillgängliga materialet. Speciellt fallstudier kan vara vinklade i något visst ramverks fördel, eftersom företag inte vill rapportera dåliga resultat eller misslyckade projekt. Konsulter vill inte heller erkänna att de använt sig av ifrågasättbara metoder.

\newpage

\section{Resultat}

\subsection{Gemensamma egenskaper hos ramverken}

\subsection{Styrkor och svagheter}

\subsection{Nämnvärda skillnader}
